%MARCOS RIAL DOCAMPO


%%%%%%%%%%%%%%%%%%%%%%%%%%%
%% CAPÍTULO 2. OBJETIVOS %%
%%%%%%%%%%%%%%%%%%%%%%%%%%%

\chapter{Objetivos}
En este trabajo se marcan dos objetivos principales y uno secundario o paralelo. Uno de los objetivos principales tratará el análisis de las diferentes clases de manglar del Golfo de Fonseca en Honduras. En el segundo objetivo principal se aplicarán los conocimientos adquiridos en el análisis para realizar una clasificación supervisada de imágenes de satélite de la zona.\\

El objetivo paralelo tratará el hecho de realizar este trabajo integramente con software libre, ventajas e inconvenientes encontrados y posibilidades de este tipo de tecnologías.

\section{Objetivo primero}
El objetivo principal del trabajo es el de realizar un análisis de la respuesta espectral de las diferentes especies de mangle que se encuentran en el Golfo de Fonseca

\section{Objetivo segundo}
Una vez realizado el análisis de especies se aplicarán los datos extraídos para realizar una clasificación supervisada de una imagen captada por el sensor Landsat TM

\section{Objetivo secundario}
A lo largo de todas las fases del presente trabajo se analizará el uso de software libre y las posibilidades que este presenta de cara a futuros proyectos. Estas fases son: redacción del documento, tratamiento, análisis y presentación de los datos así como el tratamiento y procesado de las imágenes. Se presentarán las virtudes y defectos de este tipo de software y se analizarán los problemas a medida que estos se van presentando.