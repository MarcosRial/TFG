%MARCOS RIAL DOCAMPO

\chapter*{Resumo/Resumen/Abstract}
\markboth{}{RESUMEN}
\section*{Resumo}
A proliferación da acuicultura, coa conseguinte creación de estanques de cría de camarón, son unha das principais causas da deforestación do manglar que se debe coñecer en maior profundidade. Analizouse o grao de separabilidade espectral entre especies de mangle (\textit{Rizophora mangle}, \textit{Laguncularia racemosa} y \textit{Avicennia germinans}) no Golfo de Fonseca, situado na costa pacífica de El Salvador, Honduras e Nicaragua.

Para a análise de separabilidade espectral aplicáronse funcións escritas en R de Índice de Acordo Espectral (IAE), \textit{Continuum Removal} (CR) e Ángulo Espectral a observacións de reflectividade tomados cun espectro-radiómetro portátil. Para elo creáronse en R scripts para cada un dos análises así como para a creación das gráficas de firmas espectráis. Posteriormente realizáronse diversas clasificacións de imaxes obtidas polo sensor OLI do satélite Landsat 8, tanto supervisadas coma non supervisadas, incluindo o procesado das imaxes da zona de estudio e a aplicación de índices de vexetación (NDVI e SAVI) todo isto co software GRASS GIS, que deixan claro os lugares onde se asienta o ecosistema manglar.

A análise mostra pouca separabilidade entre as especies analizadas. O resultado da análise de separabilidade espectral non marca a mellora da clasificación tal e como se esperaba nun primeiro momento.

\noindent\textbf{Palabras clave}: manglar, Golfo de Fonseca, separabilidade espectral, técnicas de análise espectral, R, reflectividade, Landsat 8, clasificación, GRASS GIS, software libre.

\section*{Resumen}
La proliferación de la acuicultura, con la consiguiente creación de estanques de cría de camarón, son una de las principales causas de la deforestación del manglar que conviene conocer en mayor profundidad. Se analizó el grado de separabilidad espectral entre especies de mangle (\textit{Rizophora mangle}, \textit{Laguncularia racemosa} y \textit{Avicennia germinans}) en el Golfo de Fonseca, situado en la costa pacífica de El Salvador, Honduras y Nicaragua.

Para el análisis de separabilidad espectral se aplicaron funciones escritas en R de Índice de Acuerdo Espectral (IAE), \textit{Continuum Removal} (CR) y Ángulo Espectral a observaciones de reflectividad tomados con un espectro-radiómetro portátil. Para ello se crearon en R scripts para cada uno de los análisis así como para la creación de las gráficas de firmas espectrales. Posteriormente se realizaron diversas clasificaciones de imágenes obtenidas por el sensor OLI del satélite Landsat 8, tanto supervisadas como no supervisadas, incluyendo el procesado de las imágenes de la zona de estudio y la aplicación de índices de vegetación (NDVI y el SAVI) todo ello con el software GRASS GIS, que dejan claro los lugares donde se asienta el ecosistema manglar.

El análisis muestra poca separabilidad entre las especies analizadas. El resultado del análisis de separabilidad espectral no marca la mejora de la clasificación tal y como se esperaba en un primer momento.

\noindent\textbf{Palabras clave}: manglar, Golfo de Fonseca, separabilidad espectral, técnicas de análisis espectral, R, reflectividad, Landsat 8, clasificación, GRASS GIS, software libre.

\section*{Abstract}
Abstract