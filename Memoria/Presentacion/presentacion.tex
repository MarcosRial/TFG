%MARCOS RIAL DOCAMPO
%Presentación en diapositivas del TFG

\documentclass[12pt]{beamer}
\usetheme{Madrid}
\usepackage[utf8]{inputenc}
\usepackage[spanish]{babel}
\usepackage{amsmath}
\usepackage{amsfonts}
\usepackage{amssymb}
\usepackage{graphicx}

\author{Marcos Rial Docampo}
\title[Análisis de Separabilidad Espectral]{ANÁLISIS DE SEPARABILIDAD ESPECTRAL DE ESPECIES DE MANGLE EN EL GOLFO DE FONSECA. APLICACIÓN A A CLASIFICACIÓN DE IMÁGENES LANDSAT}
\subtitle{Trabajo Fin de Grado}
\setbeamercovered{transparent} 
\setbeamertemplate{navigation symbols}{} 
\logo{} 
\institute[USC-EPS]{Universidad de Santiago de Compostela\\Escuela Politécnica Superior de Lugo} 
\date{\today} 
%\subject{}

\AtBeginSection{ 
\begin{frame} 
  \frametitle{Índice}
  \tableofcontents[currentsection]
\end{frame}
}

\AtBeginSubsection{ 
\begin{frame}
  \frametitle{Índice}
  \tableofcontents[currentsection,currentsubsection]
\end{frame}
}

\begin{document}

\begin{frame}
\titlepage
\end{frame}

\begin{frame}
\frametitle{Índice}
\tableofcontents
\end{frame}

\section{Introducción}
\subsection{Marco Global}
\begin{frame}
\frametitle{Marco Global}
\begin{block}{Conceptos}
\textbf{Mangle}: Especie forestal que crece en el ecosistema manglar o bosque de mangle\\
\textbf{Manglar}: Ecosistema medioambiental propio de zonas costeras tropicales
\end{block}
\end{frame}

\subsection{Objetivos}
\begin{frame}
\frametitle{Otra cosa}
Otra cosa.
\end{frame}

\section{Materiales y Métodos}
\subsection{Materiales}
\begin{frame}
material
\end{frame}
\end{document}