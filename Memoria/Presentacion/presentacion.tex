%MARCOS RIAL DOCAMPO
%Presentación en diapositivas del TFG

\documentclass[12pt]{beamer}
\usetheme{Madrid}
\usepackage[utf8]{inputenc}
\usepackage[spanish]{babel}
\usepackage{amsmath}
\usepackage{amsfonts}
\usepackage{amssymb}
\usepackage{graphicx}

\author[Marcos Rial Docampo]{Marcos Rial Docampo\\
\footnotesize{Tutores:\\
Eduardo Corbelle Rico\\
Rafael Enrique Corrales Andino}}
\title[Análisis de Separabilidad Espectral]{ANÁLISIS DE SEPARABILIDAD ESPECTRAL DE ESPECIES DE MANGLE EN EL GOLFO DE FONSECA. APLICACIÓN A A CLASIFICACIÓN DE IMÁGENES LANDSAT}
\subtitle{Trabajo Fin de Grado}
\setbeamercovered{transparent} 
\setbeamertemplate{navigation symbols}{} 
\logo{} 
\institute[USC-EPS]{Universidad de Santiago de Compostela\\Escuela Politécnica Superior de Lugo} 
\date{\today} 
%\subject{}

\AtBeginSection{ 
\begin{frame} 
  \frametitle{Índice}
  \tableofcontents[currentsection]
\end{frame}
}

\AtBeginSubsection{ 
\begin{frame}
  \frametitle{Índice}
  \tableofcontents[currentsection,currentsubsection]
\end{frame}
}

\begin{document}

\begin{frame}
\titlepage
\end{frame}

\begin{frame}
\frametitle{Índice}
\tableofcontents
\end{frame}

\section{Introducción}
\subsection{Marco Global}
\begin{frame}
\frametitle{Marco Global}
\begin{block}{Conceptos}
\textbf{Mangle}: Especie forestal que crece en el ecosistema manglar o bosque de mangle\\
\textbf{Manglar}: Ecosistema medioambiental propio de zonas costeras tropicales
\end{block}
\end{frame}

\begin{frame}
\frametitle{Marco Global}
\begin{itemize}
\item Sistema medioambiental extenso y complejo
\item Situado en zona intermareal de zonas tropicales y subtropicales
\item Compuesto por más de 80 especies forestales y 2000 animales
\item Dependiente de procesos externos
\item Ecosistema gravemente amenazado
\end{itemize}
\end{frame}

\subsection{Objetivos}
\begin{frame}
\frametitle{Objetivos. Objetivo General}
Evaluar la posibilidad de emplear imágenes multiespectrales de satélite para diferenciar distintas especies de mangle del Golfo de Fonseca.

La respuesta espectral de las diferentes especies es lo suficientemente diferente como para permitir el uso de estas imágenes.
\end{frame}

\begin{frame}
\frametitle{Objetivos. Objetivos específicos}
Análisis de separabilidad espectral de las especies:
\begin{itemize}
\item \textit{Rizophora mangle} o mangle rojo
\item \textit{Laguncularia racemosa} o mangle blanco
\item \textit{Avicennia germinans} o mangle prieto
\end{itemize}
Realizar una clasificación de imágenes de Landsat 8\\
Estudiar el empleo de software libre
\end{frame}

\subsection{Zona de estudio}
\begin{frame}
\frametitle{Zona de estudio}
Entrante natural del Golfo de Fonseca\\ (latitudes 13,595824ºN-12,757449ºN\\ longitudes 87,994615ºW-87,115709ºW)
\end{frame}

\end{document}