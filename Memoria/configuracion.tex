%MARCOS RIAL DOCAMPO
%Parte del documento principal TFG

%%%%%%%%%%%%%%%%%%%%%%%%%%%%%%%%%%%%%%%%%%%%%%%%%%%%%%
%% ARCHIVO DE CONFIGURACIÓN DEL DOCUMENTO PRINCIPAL %%
%%%%%%%%%%%%%%%%%%%%%%%%%%%%%%%%%%%%%%%%%%%%%%%%%%%%%%


\usepackage{amsmath} %Paquetes de matemáticas
\usepackage{graphicx} %Permite añadir imágenes
\usepackage{subfig} %Permite añadir varias imágenes en una misma figura
\usepackage{anysize} %Permite modificar los márgenes

\usepackage[utf8]{inputenc} %Codificación
\usepackage[spanish]{babel} %Idioma español
\usepackage[T1]{fontenc}
\usepackage[bitstream-charter]{mathdesign}
\usepackage{textcomp}
%\usepackage{times} %Fuente
\usepackage{moreverb} %Opciones adicionales de verbatim.
\usepackage{verbatim} %Texto sin formato

\usepackage{multirow} %Para poder centrar verticalmente el contenido de las celdas de una tabla
\usepackage{fancyhdr}
\usepackage{url} %Para introducir URLs
\usepackage{latexsym} %Para introducir el símbolo de LaTeX
\usepackage[colorlinks,linktocpage=true,citecolor=blue,linkcolor=blue]{hyperref} %Vínculos
\usepackage{emptypage}
\usepackage{acronym}
%\usepackage{tocbibind} %Índice en cada capítulo
\usepackage{natbib} %Bibliografía
\usepackage{caption} %Para los pies de foto

\usepackage[toc,page]{appendix}

\usepackage{booktabs}

\usepackage[paper=A4,pagesize]{typearea}
\usepackage{afterpage}
\usepackage{pdfpages} %Incrustar páginas en pdf

%\usepackage{csvsimple} %Cargar csv en forma de tabla de latex

\usepackage{listings}
\renewcommand{\lstlistingname}{Función}
\lstset{
	frame=tb,
    framerule=0pt,
    aboveskip=3mm,
    belowskip=3mm,
    framextopmargin=3pt,
    framexbottommargin=3pt,
    %framexleftmargin=0.2cm,
    framesep=0pt,
    rulesep=.4pt,
    backgroundcolor=\color{gray97},
    rulesepcolor=\color{black},
    stringstyle=\color{mauve},
    showstringspaces = false,
    basicstyle=\footnotesize\ttfamily,
    commentstyle=\color{dkgreen},
    keywordstyle=\color{blue},
    numbers=left,
    numbersep=-6.5pt,
    numberstyle=\tiny\color{gray},
    numberfirstline = false,
    breaklines=true,
    morekeywords={*,...}
   }

\usepackage{color}
\definecolor{gray97}{gray}{.97}
\definecolor{gray75}{gray}{.75}
\definecolor{gray45}{gray}{.45}
\definecolor{mauve}{rgb}{0.58,0,0.82}
\definecolor{dkgreen}{rgb}{0,0.6,0}

%Define el formato de los encabezados y pies de página del índice y la lista de acrónimos
\lhead[{\small \rightmark}]{{\small %CAPÍTULO \thechapter.
 \leftmark}}
\chead[]{}
\rhead[{\small %CAPÍTULO \thechapter. 
\leftmark}]{{\small \rightmark}}
\renewcommand{\headrulewidth}{0.5pt}

% aqui definimos el pie de pagina de las paginas pares e impares.
\lfoot[]{}
\cfoot[{\small \thepage}]{{\small \thepage}}
\rfoot[]{}
\renewcommand{\footrulewidth}{0.5pt}

% aqui definimos el encabezado y pie de pagina de la pagina inicial de un capitulo.
\fancypagestyle{plain}{
\fancyhead[L]{}
\fancyhead[C]{}
\fancyhead[R]{}
\fancyfoot[L]{}
\fancyfoot[C]{\thepage}
\fancyfoot[R]{}
\renewcommand{\headrulewidth}{0pt}
\renewcommand{\footrulewidth}{0.5pt}
}



%\marginsize{2.5cm}{2.5cm}{2cm}{2cm} %márgenes izquierdo, derecho, superior e inferior

\linespread{1.5} %interlineado

%\parindent=3mm
%\parskip=3mm

%\usepackage[spanish]{minitoc}

%\setcounter{minitocdepth}{3} 
%\setlength{\mtcindent}{0pt} 
%\renewcommand{\mtcfont}{\small\bf} 
%\renewcommand{\mtcSfont}{\small\bf}

%\hyphenation{}

\newcommand{\Sep}{\vspace{1.5em}}
\newcommand{\SmallSep}{\vspace{0.5em}}

\renewcommand{\labelitemi}{-}

%\pretolerance=2000   %activar para evitar cortes de palabras
%\tolerance=1000      %activar para evitar cortes de palabras