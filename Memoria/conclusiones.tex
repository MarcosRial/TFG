%MARCOS RIAL DOCAMPO
%Parte del documento principal TFG

%%%%%%%%%%%%%%%%%%
%% CONCLUSIONES %%
%%%%%%%%%%%%%%%%%%


\chapter{Conclusiones}

La primera conclusión que se puede extraer de la realización de este \ac{TFG} es la referente a los datos tomados en campo. Podemos concluir que para la obtención de firmas espectrales para realizar bibliotecas espectrales tomar solamente una observación de cada especie se trate de una metodología correcta. Pero a la hora de realizar un análisis de separabilidad espectral fiable y hacer el posterior volcado de los resultados del análisis a la clasificación de imágenes satélite, resultaba óptimo tener numerosas observaciones de una misma especie. De este modo se permitiría conocer más a fondo aspectos de cada especie como desviación estándar de las observaciones con un valor central para cada banda que serían datos esenciales al hacer una clasificación supervisada de las imágenes Landsat.\Sep

Otro aspecto a tener en cuenta es el del software. Durante las distintas fases de realización del trabajo resultaron llamativas las numerosas actualizaciones que recibió el software. Y aunque la mayoría fueron actualizaciones menores que no afectaban a funciones importantes del programa, cabe destacar que el cambio de versión no resultó un problema de compatibilidad de archivos provenientes de versiones anteriores. La única excepción fue el caso de GRASS, que durante la realización del \ac{TFG} liberó la versión estable 7.0 solo presentando problemas de traducción haciendo confuso su uso en algunos casos y por eso se prescindió de dicha actualización.\Sep
