%MARCOS RIAL DOCAMPO
%Parte del documento principal TFG

%%%%%%%%%%%%%%%%%%
%% CONCLUSIONES %%
%%%%%%%%%%%%%%%%%%


\chapter{Conclusiones}

En el análisis de separabilidad espectral de las tres especies de mangle resultó que dos de ellas, \textit{R. mangle} y \textit{A. germinans} resultaron muy similares, lo que se evidenciaba desde un principio por el aspecto gráfico de la respuesta espectral. Por contrario, la especie \textit{L. racemosa} mostró mayor separabilidad ante las dos especies restantes sobre todo con la \textit{R. mangle}.

Se utilizó el sensor del satélite Landsat 8. Un sensor con una resolución espacial media (30 metros) que resultó no ser adecuado para la clasificación de las especies de mangle en las imágenes disponibles. Un sensor con mayor resolución espacial y espectral unido a un espectrómetro de campo con mayor resolución espectral darían presumiblemente un mejor resultado en cuanto a las clasificaciones temáticas. La principal ventaja de utilizar estas imágenes fue la de su disponibilidad libre y gratuita.

Debido a la baja separabilidad mostrada en el análisis no fue conveniente extraer los datos necesarios para aplicarlos a una clasificación efectiva de las imágenes Landsat. Exceptuando el caso del \ac{SAM} en el que sí se aplicaron los valores medios por banda de las especies de mangle. En todo caso, lo que sí se muestra en las clasificaciones es una alta detección del bosque de mangle diferenciándose fácilmente de otras especies forestales y otras coberturas, así como la sencilla interpretación visual de los estanques de camarón y salinas.

En cuanto a la utilización de software libre este resultó plenamente efectivo en todas las fases del proyecto. Los pocos problemas surgidos fueron rápidamente solucionados bien por la comunidad seguidora del software que presentaba el problema o por los propios manuales del programa. Antes de la impresión del documento se procedió al testeo en el sistema operativo Windows, con el mismo software utilizado hasta el momento, corrigiendo elementos que entraban en conflicto asegurando así la interoperabilidad entre el funcionamiento del software en sistema operativo privado y libre.